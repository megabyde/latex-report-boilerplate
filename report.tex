\documentclass[a4paper, 12pt, numbers=enddot]{scrartcl}

\defaulthyphenchar=127
\usepackage{cmap}
\usepackage[T2A, T1]{fontenc}
\usepackage[utf8]{inputenc}

\usepackage[english, russian]{babel} 
\usepackage{indentfirst}  

\usepackage[left=3cm, right=2cm, top=1.8cm, bottom=2cm]{geometry}   
\usepackage{microtype}
\usepackage{setspace}
\onehalfspacing

\setlength{\footskip}{1 cm}
\setlength{\parindent}{1 cm}

\usepackage[section]{placeins}
\usepackage{subfigure}

%% Set font for table and figure labels
\setkomafont{captionlabel}{\small\sffamily}
\setkomafont{caption}{\small\sffamily}
\setcapindent{0pt}

%% Set label format for tables and figures
\usepackage[justification=centering]{caption}
\renewcommand*{\captionformat}{}
\DeclareCaptionFormat{figformat}{#1#2.~#3}
\DeclareCaptionFormat{tabformat}{\hfill#1#2\par#3\par}
\captionsetup[table]{textfont = bf, format = tabformat}
\captionsetup[figure]{format = figformat}

%% Set footnote format
\deffootnote[1.5em]{0em}{1em}{\textsuperscript{\thefootnotemark}\,}

%% Disable warning about deprecated \float@addtolist macro used by listings
\usepackage{scrhack}

\usepackage{tikz}
\usetikzlibrary{shapes, arrows, positioning}

\usepackage{colortbl, booktabs}
\usepackage{amsmath, amssymb}
\usepackage{enumerate, xspace, setspace, calc}

\usepackage[
  unicode, 
  colorlinks, 
  linkcolor=blue,
  citecolor=red,
  bookmarksnumbered=true,
  pdftitle={},
  pdfauthor={}
]{hyperref}

% ----------------------------------------------------------------------------------
\begin{document}

\title{\Large Анализ представлений\\ об~эффективности нововведений}
\subtitle{\normalsize в <<Правилах проведения проверки инвестиционных проектов
на предмет эффективности использования средств федерального бюджета>>}

\author{\large Иванов Константин Константинович}
\date{\large \today}
\maketitle

% ----------------------------------------------------------------------------------
\section{Введение}
В данной работе проводится анализ представлений об эффективности нововведений в
<<Правилах проведения проверки инвестиционных проектов на предмет эффективности
использования средств федерального бюджета>>, утверждённых постановлением
Правительства Российской Федерации от 12 августа 2008 г.

% ----------------------------------------------------------------------------------
\section{Обзор документа}
Анализируемый документ устанавливает порядок проведения проверки инвестиционных
проектов, предусматривающих строительство, реконструкцию и техническое
перевооружение объектов капитального строительства и (или) осуществление иных
инвестиций в основной капитал, финансируемых полностью или частично за счёт
федерального бюджета, на предмет эффективности использования выделяемых
государством капительных вложений.

Целью проведения такой проверки является оценка соответствия инвестиционного
проекта установленным качественным и количественным критериям и предельному
(минимальному) значению интегральной оценки эффективности использования средств
федерального бюджета, направляемых на капитальные вложения в целях реализации
указанного проекта. Приведён перечень оценочных критериев.

Правила проведения проверки не распространяются на инвестиционные проекты: 
\begin{itemize}
  \setlength{\itemsep}{0 pt}
  \setlength{\parsep}{0 pt} 
  \item финансируемые за счёт бюджетных ассигнований Инвестиционного фонда РФ
    (включая инвестиционные проекты, для разработки проектной
     документации которых предоставляются указанные ассигнования)
  \item реализуемые в соответствии с концессионными соглашениями, а также по
    которым приняты до 1 января 2009 г. акты Правительства РФ, либо акты
    главных распорядителей средств федерального бюджета в порядке,
    установленном Правительством РФ.
\end{itemize}

% ----------------------------------------------------------------------------------
\section{Анализ документа}
Показатель эффективности "--- это, операционально, отношение эффекта к затратам
на достижение этого эффекта. Если эффект и затраты выражены количественно, то
эффективность называется количественной. Если эффект и затраты выражены
качественно, то эффективность называется качественной. Критерий (\textit{греч.}
kriterion "--- признак для суждения) же есть не что иное, как ограничение на
значение показателя, правило, в соответствии с которым осуществляется выбор
альтернативы из множества альтернатив.

Поскольку объекты капитального строительства являются объектами развития, а
показатели эффективности являются одним из инструментов выработки решений,
данные Правила и введённые в них показатели эффективности служат для выработки
решений по развитию.

Введение качественных критериев эффективности является действительно
необходимым в рассматриваемом документе, поскольку количественные критерии
эффективности непригодны для выработки решений по развитию, поскольку не
учитывают затрат (ограничений возможностей) всех субъектов, интересы которых
затрагивает данное нововведение. Введение количественных показателей
эффективности по этой же причине является избыточным. Однако предстоит выяснить
действительно ли заявленные в Правилах <<качественные критерии оценки
эффективности>> являются таковыми.

\begin{table}
  \centering\small
  \caption{Соответствие сущностных элементов и~графических изображений логико-операционной модели}
  \label{t:Legend}

  \begin{tabular}{p{22em}p{19em}}
    \toprule
    \textsf{Сущностные элементы} & \textsf{Графический образ} \\
    \midrule
    Процессы (этапы, процедуры, операции), в~которых не осуществляется выбор. &
	  Вытянутый прямоугольник (жирные линии) с~названием процесса (по
	  центру прямоугольника) \\
    Исполнители процедур, операций & Надпись слева внутри прямоугольника \\
    Документы & Менее вытянутый прямоугольник (тонкие линии) с~названием
	  документа внутри \\
    Связи между процессами (этапами, процедурами, операциями)
	  и~документами"=входами & Прямая линия со стрелкой на~конце \\
    Связи между процессами (этапами, процедурами, операциями)
	  и~документами"=выходами & Прямая линия без стрелки на~конце \\
    Источники внешних документов"=входов & Кружок (в начале прямой линии)
	  с~названием источника над кружком \\
    Решатель (логическое условие) & Ромбик с~указанием сути условия \\
    \bottomrule
  \end{tabular}
\end{table}

% ----------------------------------------------------------------------------------
\section{Выводы}
На основе проведённого анализа устанавливаем, что Правила определяют
качественную эффективность нововведения как совокупность 9 отдельных факторов,
6~из которых характеризуют ситуацию, в которой оценивается проект, один фактор
"--- критерий в форме эффекта (необходимость строительства), один фактор "---
детализирующий и один фактор "--- обоснование частных технических решений путём
сравнения с аналогами.

\end{document}
